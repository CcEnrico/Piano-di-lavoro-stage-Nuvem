%----------------------------------------------------------------------------------------
%				Creato da Mich
%   Updated by Simone Pessotto 04/08/2015
%   Updated by Riccardo Montagnin 13/04/2017
%   Updated by Federico Silvio Busetto 15/09/2017
%----------------------------------------------------------------------------------------

\documentclass[11pt,a4paper]{article} 

\usepackage[T1]{fontenc}
\usepackage[utf8x]{inputenc}
\usepackage[italian]{babel}
\usepackage{helvet}
\usepackage{multirow}
\renewcommand*\familydefault{\sfdefault}
\linespread{1.2}
\usepackage{fancyhdr}
\usepackage{lastpage}
\usepackage{graphicx}
\usepackage{tabularx}

% Comando per le firme, per aggiungere dei campi Data, decommentare le due righe sotto
\newcommand*{\SignatureAndDate}[1]{%
    \par\noindent\makebox[4.0in]{\hrulefill} %\hfill\makebox[2.0in]{\hrulefill}%
    \par\noindent\makebox[4.0in][l]{#1}      %\hfill\makebox[2.0in][l]{Date}%
}%

\usepackage[
a4paper,
top=2.5cm,
bottom=2.5cm,
left=1.5cm,
right=1.5cm,
head=30pt,
textheight=8in,
footskip=10pt
]{geometry}

\usepackage[italian]{isodate}
\usepackage{arydshln}
\usepackage{hyperref}
\usepackage{amsfonts, amsmath, amsthm, amssymb}

% Specifies custom hyphenation points for words or words that shouldn't be hyphenated at all
\hyphenation{ionto-pho-re-tic iso-tro-pic fortran} 
\hypersetup{
    colorlinks=true,
    linkcolor=black,
    urlcolor=blue
}

%----------------------------------------------------------------------------------------
%	PAGE STYLE
% ---------------------------------------------------------------------------------------
\pagestyle{fancy}
\fancyhf{}
\setlength{\headheight}{2cm} 

% Delete the paragraph indentation
\setlength{\parindent}{0pt}

%----------------------------------------------------------------------------------------
%	UNIPD STYLE
% ---------------------------------------------------------------------------------------
\newcommand{\stileUNIPD}{
    
    % header
    \lhead{
       \textline[t]{\includegraphics[width=1cm, keepaspectratio=true]{img/UniPd.png}}
        {Piano di lavoro stage presso \\\ragioneSocAzienda}
        {\nomeStudente \cognomeStudente\\ (\matricolaStudente)}
    }

    % footer
    \rfoot{\thepage/\pageref{LastPage}} %per le prime pagine: mostra solo il numero romano
    \cfoot{}

}

%----------------------------------------------------------------------------------------
%	COMPANY STYLE
% ---------------------------------------------------------------------------------------

\newcommand{\stileAziendale}{
    % Header
    \lhead{
        \textline[t]{\includegraphics[width=1cm, keepaspectratio=true]{img/logo_azienda.png}}
        {\textbf{\ragioneSocAzienda}}
        {\sitoAzienda}
    }


    %Footer
    \lfoot{
        {\fontsize{8}{10}\selectfont
        \textbf{\ragioneSocAzienda}\\
        \indirizzoAzienda\\
        \sitoAzienda\\
    }
    }

\cfoot{{\fontsize{8}{10}\selectfont \thepage/\pageref{LastPage}}} %per le prime pagine: mostra solo il numero romano

\rfoot{{\fontsize{8}{10}\selectfont Tel. \telTutorAziendale\\
        email: \emailAzienda \\
        \partitaIVAAzienda \\}}
}

%----------------------------------------------------------------------------------------
% 	HEADING STILE
%----------------------------------------------------------------------------------------

\newcommand\textline[4][t]{%
    \noindent\parbox[#1]{.333\textwidth}{\raisebox{-0.40\height}{#2}}%
    \parbox[#1]{.333\textwidth}{\centering #3}%
    \parbox[#1]{.333\textwidth}{\raggedleft #4}%
}

% Header delle pagine UNIPD style, commentarle se si intende usare lo stile aziendale

\stileUNIPD

% Header delle pagine stile aziendale, usaree, dopo aver chiesto il consenso all'azienda
% per l'uso del logo e dei dati personali. Commentarle se si intende usare l'UNIPD style

%\stileAziendale

% Line under the heading
\renewcommand{\headrulewidth}{0.4pt}  

%-----------------------------------------------------------------------------------------
%   FOOTER STYLE
%-----------------------------------------------------------------------------------------

%***PIÈ DI PAGINA***
%\lfoot{\includegraphics[keepaspectratio = true, width = 25px] {img/UniPd.png} \textit{\nomeStudente 
%\cognomeStudente (\matricolaStudente)}\\ \footnotesize{\emailStudente}}
%\rfoot{\thepage/\pageref{LastPage}} %per le prime pagine: mostra solo il numero romano

    
\renewcommand{\footrulewidth}{0.4pt}   %Linea sopra il piè di pagina