%----------------------------------------------------------------------------------------
%	PLANNING
%----------------------------------------------------------------------------------------

\section*{Pianificazione del lavoro}

\subsection*{Pianificazione settimanale}
% \prospettoSettimanale

La pianificazione delle attività seguirà un approccio Agile, con revisione settimanale degli obiettivi e ridefinizione dei task in base all’avanzamento e alle priorità del team.

Nelle prime settimane di stage, lo studente sarà coinvolto in attività di:
\begin{itemize}
    \item \textbf{Onboarding tecnico e funzionale}, per comprendere il contesto progettuale e ambientarsi negli strumenti di sviluppo;
    \item \textbf{Studio dell’architettura esistente}, al fine di familiarizzare con le componenti software già realizzate o in fase di sviluppo;
    \item \textbf{Sperimentazione delle tecnologie adottate}, con esercitazioni pratiche su MudBlazor, ASP.NET Core, MongoDB, .NET Aspire e Docker.
\end{itemize}

Successivamente, lo studente parteciperà attivamente allo sviluppo del sistema, con particolare attenzione a:
\begin{itemize}
    \item Implementazione delle interfacce front-end in MudBlazor;
    \item Realizzazione delle logiche di back-end in ASP.NET Core;
    \item Integrazione con il database MongoDB;
    \item Gestione e validazione delle formule fisiche e dei calcoli numerici;
    \item Deployment e gestione dell’infrastruttura tramite Docker;
    \item Documentazione tecnica del progetto.
\end{itemize}

La pianificazione sarà flessibile e iterativa: le attività non verranno svolte in modo strettamente sequenziale, ma saranno distribuite in parallelo. In particolare, sviluppo, interazione con il team e stesura della documentazione avverranno in modo continuo lungo tutto il percorso.

\newpage

\subsection*{Ripartizione ore}

La pianificazione, in termini di quantità di ore di lavoro, sarà così distribuita:

\begin{center}
\begin{tabular}{|c|p{10cm}|}
    \hline
    \textbf{Ore} & \textbf{Attività} \\
    \hline
    60 & Formazione iniziale, onboarding e studio tecnologie (MudBlazor, ASP.NET Core, MongoDB, .NET Aspire, Docker) \\
    \hline
    150 & Attività di sviluppo (interfacce, logiche di configurazione, calcolo, integrazione, gestione ambienti) \\
    \hline
    70 & Partecipazione alle attività del team (sprint, daily meeting, code review, incontri col cliente, task Agile) \\
    \hline
    30 & Documentazione tecnica, demo finale e attività di chiusura dello stage \\
    \hline
    \textbf{Totale ore: 310} & \\
    \hline
\end{tabular}
\end{center}

Le attività saranno svolte in parallelo e non seguiranno un ordine cronologico rigido, in modo da adattarsi all’organizzazione Agile del team.

\newpage
