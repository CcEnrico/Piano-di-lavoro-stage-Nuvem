\section*{Contenuti formativi previsti}
% Personalizzare indicando le tecnologie e gli ambiti di interesse dello stage
Durante questo progetto di stage, lo studente avrà occasione di approfondire le sue conoscenze nell'ambito dello sviluppo di applicazioni web moderne e scalabili, con particolare attenzione ai seguenti strumenti, tecnologie e metodologie:

\begin{itemize}
    \item \textbf{MatBlazor}: libreria UI per Blazor basata su Material Design, che consente la creazione di interfacce moderne utilizzando esclusivamente C\#;
    
    \item \textbf{.NET Aspire Telemetry}: strumento per l’osservabilità di applicazioni .NET in ambienti cloud-native e architetture a microservizi;
    
    \item \textbf{.NET / C\#}: linguaggio e framework di sviluppo multipiattaforma per la realizzazione di applicazioni web, API e desktop;
    
    \item \textbf{MongoDB}: database NoSQL orientato ai documenti, adatto alla gestione di dati dinamici, con elevata flessibilità e scalabilità;
    
    \item \textbf{Testing dinamico front-end}: realizzazione di test interattivi sull’interfaccia utente sviluppata con MatBlazor, seguendo i principi del Test-Driven Development (TDD);
    
    \item \textbf{Approccio Agile}: applicazione di metodologie agili nella gestione del progetto, suddivisione del lavoro in sprint e partecipazione alle principali cerimonie Scrum;
    
    \item \textbf{Best practice di ingegneria del software}: progettazione del codice secondo principi solidi, documentazione tecnica e attenzione alla manutenibilità e qualità del software prodotto.
\end{itemize}

\newpage
