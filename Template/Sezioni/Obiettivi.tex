%----------------------------------------------------------------------------------------
%	OBJECTIVES
%----------------------------------------------------------------------------------------
\section*{Obiettivi}

\subsection*{Notazione}
Si farà riferimento ai requisiti secondo le seguenti notazioni:
\begin{itemize}
	\item \textit{O} per i requisiti obbligatori, vincolanti in quanto obiettivo primario richiesto dal committente;
	\item \textit{D} per i requisiti desiderabili, non vincolanti o strettamente necessari, ma dal riconoscibile valore aggiunto;
	\item \textit{F} per i requisiti facoltativi, rappresentanti valore aggiunto non strettamente competitivo.
\end{itemize}

Le sigle precedentemente indicate saranno seguite da una coppia sequenziale di numeri, identificativo del requisito.

\subsection*{Obiettivi fissati}
Si prevede lo svolgimento dei seguenti obiettivi:

\begin{itemize}
	\item Obbligatori
	\begin{itemize}
		\item O.01 Realizzare un’applicazione web per la configurazione dei pannelli della linea Automation destinati al controllo delle valvole industriali;
		\item O.02 Automatizzare il calcolo del costo finale della configurazione sulla base dei componenti selezionati;
		\item O.03 Applicare vincoli dinamici sui componenti selezionabili, in base al tipo di cliente e all’ambiente di installazione previsto;
		\item O.04 Velocizzare la generazione di preventivi, riducendo il tempo impiegato nella fase di offerta grazie alla sostituzione dell’attuale procedura manuale.
	\end{itemize}
	
	\item Desiderabili 
	\begin{itemize}
		\item D.01 Favorire l’adozione dell’applicativo da parte del team tecnico-commerciale attraverso un’interfaccia intuitiva e moderna.
	\end{itemize}
	
	\item Facoltativi
	\begin{itemize}
		\item Nessuno obiettivo facoltativo è stato definito in questa fase.
	\end{itemize} 
\end{itemize}

\newpage
