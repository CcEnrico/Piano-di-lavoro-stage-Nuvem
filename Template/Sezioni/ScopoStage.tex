%----------------------------------------------------------------------------------------
%	STAGE DESCRIPTION
%----------------------------------------------------------------------------------------
\section*{Scopo dello stage}

Lo scopo di questo progetto di stage è la realizzazione di un'applicazione web per la configurazione dinamica di pannelli da installare su valvole industriali, con calcolo automatico del costo finale basato sulla componentistica selezionata. Ogni pannello potrà essere personalizzato con diverse tipologie di componenti specializzate, e il sistema applicherà logiche di calcolo e vincoli tecnici per produrre un output validato e coerente.

Lo studente avrà il compito di partecipare allo sviluppo completo del sistema, che comprende:

\begin{itemize}
    \item Front-end realizzato con \textbf{MudBlazor};
    \item Back-end sviluppato con \textbf{ASP.NET Core};
    \item Persistenza dei dati tramite \textbf{MongoDB};
    \item Osservabilità e orchestrazione tramite \textbf{.NET Aspire};
    \item Gestione dell’infrastruttura e del deployment tramite \textbf{Docker}.
\end{itemize}

Lo stage sarà svolto all’interno del team di sviluppo che opera per il cliente \textbf{Starline Services S.p.A.}, in un contesto strutturato e professionale. Lo studente potrà apprendere le dinamiche di un team consolidato, le \textit{best practice} di ingegneria del software e le metodologie adottate per la realizzazione e manutenzione di software enterprise.

Nonostante lo stage copra solo una parte della durata complessiva del progetto, sarà sufficiente per offrire un’immersione significativa nel ciclo di vita del software, con particolare attenzione alla progettazione, sviluppo, validazione e gestione in ambienti on-premise.
