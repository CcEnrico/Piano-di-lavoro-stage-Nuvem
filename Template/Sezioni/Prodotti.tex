%----------------------------------------------------------------------------------------
%	DESCRIPTION OF THE PRODUCTS THAT ARE BEING EXPECTED FROM THE STAGE
%----------------------------------------------------------------------------------------
\section*{Prodotti attesi}

Lo studente dovrà realizzare un'applicazione web rivolta al cliente \textbf{Starline Services S.p.A.}, una sottodivisione specializzata nella produzione e configurazione di valvole industriali. L'applicativo avrà lo scopo di semplificare e automatizzare la configurazione dei \textbf{pannelli della linea Automation}, destinati al controllo e all’azionamento di valvole, sostituendo l'attuale procedura manuale basata su fogli Excel complessi e soggetti a errori.

I prodotti da configurare fanno riferimento alla sezione \textbf{Products} del sito ufficiale dell’azienda, accessibile al link: \url{https://starline.it/products/}.

Il sistema dovrà permettere:

\begin{itemize}
    \item La visualizzazione di una lista di pannelli e dei relativi componenti disponibili;
    \item La selezione e configurazione personalizzata di un pannello da parte dell’utente;
    \item L'applicazione di vincoli dinamici sui componenti selezionabili, in base al tipo di cliente e all’ambiente di installazione previsto;
    \item Il calcolo automatico e validato del costo finale della configurazione;
    \item La sostituzione completa dell’attuale processo manuale con uno strumento digitale efficiente, sicuro e distribuito in ambiente \textbf{on-premise}.
\end{itemize}

L’applicazione dovrà garantire un’interfaccia moderna e responsive, una user experience fluida e una logica di configurazione solida, validata e facilmente estendibile.

\bigskip

Inoltre, lo studente dovrà produrre una relazione scritta che illustri i seguenti punti:

\begin{enumerate}
    \item \textbf{Analisi dei requisiti} \\
    Descrizione dettagliata delle funzionalità richieste, dei vincoli progettuali e delle tecnologie scelte.

    \item \textbf{Architettura del sistema} \\
    Descrizione dell’architettura software, suddivisione in moduli, flussi di dati e infrastruttura di distribuzione.

    \item \textbf{Implementazione} \\
    Dettagli sull’implementazione dell’interfaccia utente in MudBlazor, della logica di business in ASP.NET Core e del sistema di persistenza in MongoDB.

    \item \textbf{Verifica delle formule di calcolo} \\
    Validazione dei risultati numerici generati, in particolare delle formule fisiche coinvolte nella configurazione tecnica dei pannelli.
\end{enumerate}
