%----------------------------------------------------------------------------------------
%	DESCRIPTION OF THE PRODUCTS THAT ARE BEING EXPECTED FROM THE STAGE
%----------------------------------------------------------------------------------------
\section*{Prodotti attesi}
Lo studente dovrà realizzare un applicativo web che consenta:
\begin{itemize}
    \item La visualizzazione di un catalogo di prodotti;
    \item La selezione di un prodotto da parte dell’utente;
    \item La personalizzazione dinamica del prodotto tramite la scelta delle componenti disponibili;
    \item Il calcolo automatico del costo finale in base alla configurazione selezionata.
\end{itemize}

Il sistema dovrà garantire un’interfaccia moderna, una user experience fluida e intuitiva, oltre a una logica di calcolo robusta, performante e scalabile.

\bigskip

Inoltre, lo studente dovrà produrre una relazione scritta che illustri i seguenti punti:
\begin{enumerate}
    \item Analisi dei requisiti \\
    Descrizione dettagliata delle funzionalità richieste, delle tecnologie scelte e del target dell’applicativo.

    \item Architettura del sistema \\
    Descrizione dell’architettura software, suddivisione in componenti/moduli, flussi di dati e scelta del modello di sviluppo.

    \item Implementazione \\
    Dettagli sull’implementazione dell’interfaccia utente, della logica di business e del sistema di calcolo.

    \item Test e validazione \\
    Strategie di test adottate, risultati ottenuti e considerazioni sull’affidabilità e sulla scalabilità del sistema.
\end{enumerate}

% Nel caso in cui, in seguito all'analisi e allo sviluppo del progetto, lo studente abbia ancora tempo a disposizione, è auspicabile l’estensione del sistema con funzionalità aggiuntive, come ad esempio:
% \begin{itemize}
%     \item Gestione utenti con autenticazione;
%     \item Salvataggio e condivisione delle configurazioni personalizzate;
%     \item Integrazione con sistemi di pagamento o e-commerce.
% \end{itemize}
