\section*{Contenuti formativi previsti}

Durante questo progetto di stage, lo studente avrà occasione di approfondire le sue conoscenze nello sviluppo di applicazioni web moderne, scalabili e orientate alla configurazione dinamica di prodotti industriali. In particolare, approfondirà:

\begin{itemize}
    \item \textbf{MudBlazor}: libreria UI per Blazor basata su Material Design, utile per realizzare interfacce moderne e responsive completamente in C\#;
    
    \item \textbf{ASP.NET Core}: framework back-end modulare e ad alte prestazioni per la realizzazione di API e logiche applicative in ambiente .NET;
    
    \item \textbf{.NET Aspire}: piattaforma per la configurazione, osservabilità e orchestrazione di applicazioni distribuite in ambienti .NET, utile in contesti enterprise e cloud-native;
    
    \item \textbf{MongoDB}: database NoSQL orientato ai documenti, adatto alla gestione di strutture dati flessibili, complesse e dinamiche come quelle legate alla configurazione dei pannelli;
    
    \item \textbf{Docker}: tecnologia per la containerizzazione dei servizi e la gestione dell’infrastruttura, utilizzata sia in locale (per ambienti di test) sia in produzione on-premise presso il cliente;
    
    \item \textbf{Verifica delle logiche di calcolo}: implementazione e validazione delle formule fisiche e matematiche necessarie al corretto funzionamento del sistema di configurazione;
    
    \item \textbf{Approccio Agile}: partecipazione attiva al ciclo di sviluppo Agile adottato dal team, con organizzazione del lavoro in sprint e interazione costante con il team tramite cerimonie Scrum;
    
    \item \textbf{Best practice di ingegneria del software}: progettazione modulare e manutenibile, gestione del versionamento, scrittura di documentazione tecnica chiara e aggiornata.
\end{itemize}

\newpage
