%----------------------------------------------------------------------------------------
%	STAGE DESCRIPTION
%----------------------------------------------------------------------------------------
\section*{Scopo dello stage}

Lo scopo di questo progetto di stage è la realizzazione di un'applicazione web per la configurazione dinamica di pannelli per l'automazione industriale di valvole automatizzate, con calcolo automatico del costo finale basato sulla componentistica selezionata. Ogni pannello potrà essere personalizzato con diverse tipologie di componenti specializzate, e il sistema applicherà logiche di calcolo e vincoli tecnici e dimensionali per produrre un output validato e coerente.

Lo studente avrà il compito di partecipare allo sviluppo completo del sistema, che comprende:

\begin{itemize}
    \item \textbf{Blazor}: framework .NET per lo sviluppo di applicazioni web interattive, che consente di scrivere l’interfaccia in C\# senza l’uso di JavaScript;
    \item \textbf{MudBlazor}: libreria di componenti UI basata su Material Design, utilizzata con Blazor per creare interfacce moderne, coerenti e responsive;
    \item \textbf{ASP.NET Core}: tecnologia utilizzata per lo sviluppo del back-end, che gestisce la logica di business dell’applicazione;
    \item \textbf{MongoDB}: database NoSQL orientato ai documenti, scelto per la gestione flessibile dei dati relativi alle configurazioni dei pannelli;
    \item \textbf{.NET Aspire}: strumento per l’osservabilità e l’orchestrazione di servizi distribuiti in ambienti .NET;
    \item \textbf{Docker}: piattaforma per la containerizzazione e il deployment dell’applicazione in ambienti on-premise.
\end{itemize}

Lo stage sarà svolto all’interno del team di sviluppo che opera per il cliente \textbf{Starline Services S.p.A.}, in un contesto strutturato e professionale. Lo studente potrà apprendere le dinamiche di un team consolidato, le \textit{best practice} di ingegneria del software e le metodologie adottate per la realizzazione e manutenzione di software enterprise.

Nonostante lo stage copra solo una parte della durata complessiva del progetto, sarà sufficiente per offrire un’immersione significativa nel ciclo di vita del software, con particolare attenzione alla progettazione, sviluppo, validazione e gestione in ambienti on-premise.
