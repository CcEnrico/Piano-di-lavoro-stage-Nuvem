%----------------------------------------------------------------------------------------
%	STAGE DESCRIPTION
%----------------------------------------------------------------------------------------
\section*{Scopo dello stage}

Lo scopo di questo progetto di stage è la realizzazione di un'applicazione web per il calcolo dinamico del costo dei prodotti in base alle componenti selezionate dall’utente. Ogni prodotto potrà essere personalizzato con diverse tipologie di componentistica specializzata, e il sistema calcolerà il costo finale sulla base di tali scelte, con logiche non banali.

Lo studente avrà il compito di partecipare allo sviluppo completo del sistema, che comprende:

\begin{itemize}
    \item Front-end realizzato con \textbf{MatBlazor};
    \item Back-end sviluppato con \textbf{.NET / C\#};
    \item Persistenza dei dati tramite \textbf{MongoDB};
    \item Osservabilità attraverso \textbf{.NET Aspire Telemetry}.
\end{itemize}

In aggiunta, lo stage ha anche uno scopo formativo e professionale: lo studente verrà inserito all’interno di un team di sviluppo già consolidato, dove potrà apprendere le dinamiche di lavoro reali, le \textit{best practice} adottate e i principi dell’ingegneria del software applicata in un contesto aziendale strutturato.

Nonostante lo stage copra solo una parte della durata complessiva del progetto, sarà sufficiente per offrire un’immersione significativa nel flusso di lavoro e nelle metodologie utilizzate dal team.
